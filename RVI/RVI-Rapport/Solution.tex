\chapter{Solution}

Pour réaliser notre projet, nous aurons besoin d’une piscine dans laquelle sera plongé le ou les utilisateurs. Ceux-ci seront équipés d’une combinaison de plongée, un casque de réalité virtuelle ainsi que de batteries et d’une unité centrale pour communiquer avec un serveur de donnée et fournir à l’Oculus Rift les images à afficher. Ces éléments seront placés sur son dos. Comme l’utilisateur sera dans l’eau, le poids de ces éléments est moins à considérer.
 
L’utilisateur sera plongé dans une piscine équipé avec des caméras OptiTrack qui pourront tracker l’ensemble des utilisateurs et retranscrire virtuellement leurs membres. Cela permettra également de représenter les utilisateurs dans l’environnement virtuel pour éviter les collisions entre utilisateurs, puisque un ensemble d’utilisateurs seront placés dans la même piscine. 


\begin{figure}[!ht]
	\center	
	\includegraphics[scale=0.3]{image/piscine.png}
	\caption{Proposition d'intégration de OptiTrack}
\end{figure}


Concernant les bords de la piscine, ils seront représentés virtuellement sous la forme de falaise, ou autre éléments dits physique qui délimiterons les bords de son espace d’interaction.

L’avantage de cette solution est qu’elle permet à un nombre important d’utilisateur d’être immergés dans un même monde virtuel dans lequel ils pourront interagir entre eux ainsi qu’avec les mêmes éléments virtuels. On peut imaginer pour les éléments virtuels, des poissons, des fonds marins, etc.
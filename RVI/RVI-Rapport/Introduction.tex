\chapter{Introduction}
Dans le cadre de notre projet de Réalité Virtuelle et Interaction, nous avons à présenter un projet. Notre projet, que nous avons appelé Thalassa, consiste à simuler la plongée pour plusieurs utilisateurs.

Plusieurs individus seront immergés dans un bassin,avec un équipement de plongée complet. Concernant la partie réalité virtuelle, un Oculus étanche sera placé sur la tête des sujets. Des marqueurs seront disposés sur sa combinaison de plongée pour permette à des caméras de tracking infrarouge, situés en dehors du bassin, de tracker les mouvements ainsi que les membres des utilisateurs. 

Pour notre projet, nous pouvons imaginer que nous répondons à une demande. La société Nausicaa veut mettre au point une nouvelle attraction pour son parc afin de proposer à ses clients une expérience immersive de plongée sous-marine. L’attraction devra permettre au public de plonger dans l’eau tout en donnant la possibilité de nager et de visualiser un environnement sous-marin virtuel. 

Comment mettre en oeuvre la partie matériel et logiciel afin de crée cette nouvelle attraction ?

C'est ce que ce rapport va vous présenter. Dans un premier temps une recherche de l'éxistant sera présenté, suivit d'une explication conrète de notre application. Ensuite le logiciel et le matériel nécessaire seront présentés.

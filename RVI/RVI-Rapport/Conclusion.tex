\chapter{Conclusion}

Pour répondre à la demande de créer un simulateur de plongée sous-marine, le projet Thalassa propose d'immerger un groupe d’utilisateur dans une piscine. Ces utilisateurs seront équipé d’un matériel de plongée complet, d’un Oculus Rift et seront trackés par un ensemble de caméras infrarouge.


En utilisant l’Oculus Rift nous pouvons prévoir différents environnements virtuels, comme par exemple la visite d’un récif ou encore un navire sous l’eau.


Cette solution présente l’avantage de faire ce qui est pour l’instant impossible à simuler, l’eau,  en immergeant les utilisateurs. De plus dans le cadre d’une attraction disponible pour Nausicaa cette solution permettrait de faire profiter un ensemble de personne (famille) dans un temps relativement correct (temps de simulation et équipement des utilisateurs) et permet d'être accessible en termes de coûts d’équipements.
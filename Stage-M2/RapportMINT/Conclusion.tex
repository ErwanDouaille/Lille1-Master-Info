\chapter{Conclusion et perspectives}

Sur le plan technique ce stage m'a permis de découvrir l'ensemble des moyens mis en œuvre aujourd'hui pour permettre à nos systèmes d'exploitation de faire un rendu à l'écran. L'exploration du code source de Gnome, Clutter, Mutter, X ou encore Wayland m'ont permis de mener à bien ce projet.

L'utilisation de mes compétences acquises en Master 2 IVI, notamment tout ce qui concerne OpenGL, m'ont permis de réaliser ce projet. La quête de l'optimisation m'a également permis de découvrir des techniques telles que le mipmap d'OpenGL.

\section*{Perspectives}

Ce projet est un aperçu de ce qui peut être mis en œuvre pour apporter un gain en confort visuel pour l'utilisation des écrans géants. Des études mettant en œuvre ce qui a été présenté dans ce rapport doivent être réalisé pour pouvoir mettre en évidence les gains en performances de l'utilisation de ces techniques sur un grand écran.

Sur le plan technique, une réécriture de certains des fichiers est nécessaire pour permettre une meilleure intégration dans le système de base de Gnome. Sur ce sujet, nous communiquons actuellement avec la communauté Gnome pour intégrer ce travail aux versions officielles de Gnome. 
	
Sur le plan interaction, peu de choses ont étés explorées. Cependant c'est une partie essentielle notamment pour l'utilisation de la loupe. La partie interaction devra donc être complétée sur le plan technique et recherche, pour permettre une utilisation plus performante pour les grands écrans.

Une question qui n'a pas été développé dans ce rapport est la possibilité ou non d'apporter ce concept à une utilisation multi-utilisateur. Car pour l'instant la déformation ne se réalise que par le suivit d'une seule personne. La question du multi-utilisateur apporte une complexité dans la déformation de l'image, à savoir si oui ou non la formule est-elle compatible pour le multi-utilisateur, si non comment déformer l'écran et quelles parties de l'écran.
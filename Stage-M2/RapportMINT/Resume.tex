\section*{Résumé}

Dans le cadre de mon stage de fin d’étude, j’ai du proposer une solution pour améliorer le confort visuel d'utilisation d'écrans géants. J'ai effectué mon stage au sein de l’équipe MINT du laboratoire CRIStAL.
Pour proposer une amélioration visuelle de l'utilisation de grands écrans j'ai choisis d'utiliser l'anamorphose dynamique pour dilater ou compresser l'affichage en fonction de la position de l'utilisateur face a l'écran. Pour aider a visualiser a distance l'écran une loupe a également été developpée.
% ca c'est bon
Cet ensemble d'outils permet de palier à plusieurs problèmes liés aux grands écrans, tels que l'effet de perspective lors d'une utilisation rapprochée ou encore l'aide à la lecture sur une résolution d'affichage élevée.

% faire plus generique pas parler de la namorphose dynamique
% proposition d'un cadre logiciel mis en oeuvre dans 2 cas loupe + ana


\section*{Abstract}

This report describes the internship I spend at the Inria's (Institut National de Recherche en Informa-
tique et en Automatique) Mint team. The purpose of the project I lead is to find a way to bring more visual confort for the use of large screen. Dynamic anamorphosis is one of the concepts I used. This is a distorsion of the rendering screen depending of the user positions from the screen. To realised this project I used the operating system GNU/Linux with the Gnome desktop environment. For a long range use of the large screen I decided to implement some magnifying glasses which are a bit different than what modern operating systems provides. All of this tools are fixes for what large screen problems such as perspective or high DPI screens.

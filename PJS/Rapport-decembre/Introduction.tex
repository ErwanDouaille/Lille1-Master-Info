\chapter{Introduction}
Dans le cadre de notre Master 2 spécialité IVI (Image Vision Interaction), nous avons été amenés à réaliser un projet en lien avec cette formation. Le but de notre projet est de modifier une application de tracking existante afin d’en corriger certains défauts et de la rendre multi-utilisateurs. 

Cette application de tracking permet de contrôler le pointeur souris à distance grâce au tracking de la main. Elle se base sur la création d’un espace virtuel par l’utilisateur, celui-ci peut ainsi contrôler la souris avec sa main quand celle-ci se trouve dans l’espace virtuel. Cependant, cette application n’est pas complète et possède quelques défauts. Tout d’abord, nous devons corriger le fait que l’espace d'interaction n’est pas visible et est donc difficile à manipuler. Ensuite, il faut rendre cette application multi-utilisateurs. 
	
Dans une première partie, nous présenterons en détail l’application fournie, les technologies utilisées, ses défauts ainsi que les axes d’améliorations possibles. Dans une seconde partie, nous présenterons un état de l’art afin de connaître les dernières technologies existantes sur le sujet ainsi que les améliorations que nous pouvons effectuer. Enfin dans une troisième partie nous exposerons les pistes de travail que nous avons eu et quelles solutions nous avons retenu.

\chapter*{Résumé}

Nous avons travaillé sur un projet en coopération avec Hanaë Rateau, doctorante au sein de l’équipe MINT. Ce projet consiste en une application de tracking utilisant la Kinect et permettant à un utilisateur de contrôler à distance une souris à l’aide d’un espace d'interaction virtuel défini par lui-même. 
Cependant, cette application possède des défauts que nous devons corriger durant notre projet. En effet, il est difficile pour l’utilisateur de visualiser l’espace d’interaction virtuel, d’autant plus que celui-ci n’a pas de retour physique. De plus dans le cas d’ajout d’un support physique, il existerai un problème d’occlusion. Enfin, l’application n’est pas encore prévue pour être multi-utilisateurs.  
Pour commencer, nous avons effectué des recherches afin de dresser un état de l’art en rapport avec les problèmes que nous devons résoudre. Nous avons ainsi trouvé des articles portant sur le retour visuel et sensoriel, sur l’occlusion de la main et sur le tracking de support physique. Ensuite, nous avons dressé plusieurs hypothèses afin d’améliorer l’application :

\begin{itemize}
	\item à l’aide d’ARTracks
	\item utiliser des tag QR Code
	\item utiliser plusieurs Kinect
	\item utiliser une Kinect en hauteur
\end{itemize}

Au final c’est cette dernière solution que nous avons choisie, car elle possède certains avantages quant à la résolution des problèmes et est peu coûteuse et peut être plus facilement abordable pour l’utilisateur lambda, tant qu’il respecte les conditions d’installations.





\documentclass[a4paper,12pt]{report}
\usepackage[utf8]{inputenc}
\usepackage[francais]{babel}
\usepackage{fancyhdr}
\usepackage{graphicx}
\usepackage{tikz}
\usetikzlibrary{calc}
\usepackage{listings}
\usepackage{xcolor}
\definecolor{grey}{rgb}{0.9,0.9,0.9}
\usepackage{titlesec}
\usepackage{verbatim}
\usepackage{listings}
\usepackage{textcomp}
\usepackage{hyperref}
\usepackage{longtable}
\usepackage{colortbl}
\usepackage{amssymb}


\frenchbsetup{StandardLists=true}
\newcommand{\marge}{18mm}
\usepackage[left=\marge,right=\marge,top=\marge,bottom=\marge]{geometry}
\pagestyle{fancy}
\setlength{\headheight}{14pt}
\chead{
  \textbf{Binôme :} Douaille Erwan \& Yanis Nait Abdelaziz
  \hspace{2em}
  \textbf{Groupe :} M1 Info TI}
\renewcommand{\headrulewidth}{1pt}
\linespread{1}
\setlength{\columnseprule}{0.2pt}
\definecolor{javakeyword}{rgb}{0,0,0.5}
\definecolor{javastring}{rgb}{0,0.5,0}
\definecolor{javacomment}{rgb}{0.5,0.5,0.5}
\lstdefinestyle{java}{
   language=Java, basicstyle=\footnotesize,       % the size of the fonts that are used for the code
  numbers=left,                   % where to put the line-numbers
  numberstyle=\tiny\color{gray},  % the style that is used for the line-numbers
  stepnumber=1,                   % the step between two line-numbers. If it's 1, each line
                                  % will be numbered
  numbersep=5pt,                  % how far the line-numbers are from the code
  backgroundcolor=\color{white},  % choose the background color. You must add \usepackage{color}
  showspaces=false,               % show spaces adding particular underscores
  showstringspaces=false,         % underline spaces within strings
  showtabs=false,                 % show tabs within strings adding particular underscores
  frame=single,                   % adds a frame around the code
  rulecolor=\color{black},        % if not set, the frame-color may be changed on line-breaks within not-black text (e.g. commens (green here))
  tabsize=2,                      % sets default tabsize to 2 spaces
  captionpos=b,                   % sets the caption-position to bottom
  breaklines=true,                % sets automatic line breaking
  breakatwhitespace=false,        % sets if automatic breaks should only happen at whitespace
  title=\lstname,                 % show the filename of files included with \lstinputlisting;
   stringstyle=\color{javastring},
   keywordstyle=\color{javakeyword}\ttfamily\textbf,
   commentstyle=\color{javacomment}\ttfamily\textit
 }

\begin{document}



\makeatletter
\begin{titlepage}
\centering
\vspace{-10em}
{\LARGE \textbf{\textsc{Rapport de stage Master 2}}}\\
\vspace{3em}
\includegraphics[scale=0.45]{image/cover.jpg}\\
\vspace{3em}
{\LARGE \textsc{Amélioration du confort visuel sur grand écran par déformation dynamique de l'image}}\\

\vspace{8em}
Par\\
\vspace{1em}
{\LARGE \@author}\\

\vspace{2em}

\flushleft Encadrants: Laurent Grisoni et Samuel Degrande 


\begin{tikzpicture}[remember picture,overlay]

\node [below left,xshift=-1cm, yshift=4cm] at (current page.south east){\includegraphics[scale=0.6]{image/ustl1.png}};

\end{tikzpicture}
\end{titlepage}
\makeatother

\sloppy

\setcounter{page}{1} 
\newpage

\section*{Introduction}

Dans ce tp, nous allons travailler sur des images dicrètes en analysant la transformée de Fourier de celle-ci. En effet la tranformée de Fourier d'une image permet de passer d'une représentation spatiale de l'image dans le domaine fréquentiel.

\subsection*{Manipulations}

En chargeant l'image 256$\_$a.jpg nous permet d'obtenir l'image suivante
\begin{figure}[!ht]
	\center
	\includegraphics[scale=0.5]{image/256_a.jpg}
\end{figure}

Pour déterminer la période spatiale $T$ séparant deux maxima locaux consécutifs du motif cyclique il suffit de prendre deux points se trouvant au même niveau dans deux motifs blancs consécutifs:
\begin{center}
$y1$=9 \\
$y2$=14 \\
$T$=14-9=5 pixels 
\end{center}

\subsubsection*{Question 1}

La fréquence spatiale $w$ du motif cyclique correspond à l'inverse de la période spatiale $T$ calculée précédemment:
\begin{center}
$\omega$=1/$T$ \\
$\omega$=1/5 \\
$\omega$=0.2 cycles/pixel
\end{center} 

\newpage

\subsubsection*{Question 2}
En exécutant la macro qui calcule la FFT de l'image '256$\_$a.jpg' on obtient l'image suivante:

\begin{figure}[!ht]
	\center
	\includegraphics[scale=0.5]{image/question2.png}
\end{figure}

	
Les coordonnées et la valeur de la raie maximale obtenus sont:
\begin{center}
	$i$=128 \\
	$j$=128 \\
	$value$=254
\end{center}

\newpage

\subsubsection*{Question 3}
%Insérer le code de la macro ici
Les coordonnées et la valeur de la raie secondaire ainsi que la valeur de la distance $D$ séparant la raie principale de la raie secondaire obtenus sont:
\begin{center}
$i$=128 \\
$j$=77 \\
$value$=240 \\
$D$=51 pixels
\end{center}

\subsubsection*{Question 5}
La valeur de la fréquence spatiale peut être obtenu avec la formule suivante: \\
\begin{center}
	$\omega$=$D$/$H$ (Pour un motif cyclique vertical) \\
	$\omega$=$D$/$W$ (Pour un motif cyclique horizontal)
\end{center}

\subsubsection*{Question 6}
La fréquence spatiale peut donc être calculée de la façon suivante :
\begin{center}
	$\omega$=D/$\sqrt{H*W)}$
\end{center}
Pour les images de taille 256*256 on obtient le tableau de résultats suivants:

\begin{center}
\begin{longtable}[c]{|p{0.1\linewidth}| p{0.2\linewidth}| p{0.1\linewidth}|p{0.3\linewidth}|p{0.1\linewidth}|p{0.1\linewidth}|} 

	\hline
		
	\cellcolor{gray!40}\textbf{Image}	& \cellcolor{gray!40}\textbf{Direction} & \cellcolor{gray!40}\textbf{Distance}& \cellcolor{gray!40}\textbf{Fréquence sptaiale}& \cellcolor{gray!40}\textbf{Px} & \cellcolor{gray!40}\textbf{Py} \\ \hline
	
	
256$\_$a	&Verticale		&51			&0.2					&128						&77 \\ \hline
256$\_$b	&Diagonale		&72.1249		&0.28				&77						&77\\ \hline
256$\_$c	&Horizontale		&51			&0.2					&77						&128\\ \hline
256$\_$d	&Verticale		&26			&0.1					&128						&102\\ \hline
256$\_$e	&Diagonale		&108.8944	&0.43				&51						&51\\ \hline
256$\_$f	&Horizontale		&26			&0.1					&102						&128\\ \hline
256$\_$g	&Verticale		&5			&0.02				&128						&123\\ \hline
256$\_$h	&Diagonale		&7.0711		&0.03				&123						&123\\ \hline
256$\_$i	&Horizontale		&5			&0.02				&123						&128\\ \hline
	
\end{longtable}
\end{center}

 
Pour les images de taille 512*512 on obtient le tableau de résultats suivants:


\begin{center}
\begin{longtable}[c]{|p{0.1\linewidth}| p{0.2\linewidth}| p{0.1\linewidth}|p{0.3\linewidth}|p{0.1\linewidth}|p{0.1\linewidth}|} 

	\hline
		
	\cellcolor{gray!40}\textbf{Image}	& \cellcolor{gray!40}\textbf{Direction} & \cellcolor{gray!40}\textbf{Distance}& \cellcolor{gray!40}\textbf{Fréquence sptaiale}& \cellcolor{gray!40}\textbf{Px} & \cellcolor{gray!40}\textbf{Py} \\ \hline
	
	

512$\_$a	&Verticale		&102			&0.2					&256						&154\\ \hline
512$\_$b&	Diagonale	&	289.9138	&0.57				&51						&51\\ \hline
512$\_$c&	Horizontale&		102		&	0.2			&		154				&		256\\ \hline
512$\_$d	&Verticale	&	51			&0.1			&		256					&	205\\ \hline
512$\_$e	&Diagonale	&	72.1249		&0.14			&	205				&		205\\ \hline
512$\_$f	&Horizontale		&51			&0.1				&	205				&		256\\ \hline
512$\_$g	&Verticale		&10			&0.02			&	256				&		246\\ \hline
512$\_$h	&Diagonale		&14.1421		&0.03			&	246				&		246\\ \hline
512$\_$i	&Horizontale		&10			&0.02			&	246				&		256\\ \hline
	
\end{longtable}
\end{center}


\subsubsection*{Question 7}
Pour classifier les images, il suffit d'analyser les variations entre les positions x et y de la raie principale et de la raie secondaire. Si seule la valeur de x qui varie alors on a à faire à une texture horizontale; si seule la valeur de y qui varie alors on à faire à une texture verticale; et si les valeurs de x et y varient alors on a à faire à une texture diagonale.
%Insérer le code ici 

\begin{lstlisting}[float,style=Java,caption={Code question 7},label=lst:question 7]
// recherche du max (normalement localise au centre du plan de fourier
max_1 = 0; 
i_max_1 = 0;
j_max_1 = 0;

for (j=0; j<H; j++) {
   for (i=0; i<W; i++){
	p = getPixel(i,j);
	if ( max_1 < p){
		max_1 =p;
		i_max_1 = i;
		j_max_1 =j;
	} 
   }
}
// mise a zero de la valeur max
setPixel (i_max_1,j_max_1,0);
i_max_2 = 0;
j_max_2 = 0;
max_2=0;
for(j=0;j<H;j++){
	for(i=0;i<W;i++){
			p=getPixel(i,j);
			if(p>max_2){
				max_2=p;
				i_max_2 = i;
				j_max_2 =j;
			}
	}
}
distance = sqrt(pow(i_max_2 - i_max_1, 2) + pow(j_max_2 - j_max_1, 2));
if(i_max_1!=i_max_2 && j_max_1!=j_max_2){
	print("texture diagonale");
} else if(i_max_1==i_max_2 && j_max_1!=j_max_2){
	print("texture verticale");
} else if(i_max_1!=i_max_2 && j_max_1==j_max_2){
	print("texture horizontal");
}
omega=D/sqrt(W*H);
print("omega=",omega);
\end{lstlisting}




\newpage
%%%%%%%%%%%%%%%%%%%%%%%%%%%%%%%%%%%%%%%%%% CONCLU
%%%%%%%%%%%%%%%%%%%%%%%%%%%%%%%%%%%%%%%%%%%%%%%%%
%%%%%%%%%%%%%%%%%%%%%%%%%%%%%%%%%%%%%%%%%%%%%%%%%
\section*{Conclusion}
Dans ce tp , nous avons pu analyser un image discrète numérique en travaillant sur sa transformée de Fourier et non sur l'image elle même. La transformée de Fourier nous a permis en effet de déterminer la direction d'un motif cyclique contenue dans une image en étudiant les positions des raies principale et secondaire.   
\end{document}
\chapter{Scenarios}

\section{Etudiants}

\subsection{Notifications}

Les notifications préviennent les étudiants d'un événement, tel qu'un examen, un cours ou un emprunt de livre. D'autre utilisateur peuvent en recevoir, comme un professeur pour une modification de cours.

\subsection*{Notification d’examen}
Acteur : Albert (Etudiant)
\begin{itemize}
\item Le système envoie une notification à Albert, sur un examen de maths dans 30 minutes 
\item Albert ouvre la notification sur son smartphone 
\item L’application affiche “Examen en Codage en salle M5 A12 à 10h30”  
\item Albert sélectionne M5 A12 pour lui permettre de se rendre à l’examen 
\item L’application affiche la page “Plan” 
\item L’application affiche une carte indiquant le chemin à suivre pour se rendre à la salle d’examen depuis sa position actuelle 
\end{itemize}

\subsection*{Annulation de cours}
Acteur : Benoit (Etudiant)
\begin{itemize}
\item Benoit sort du restaurant universitaire, il se dirige vers le lieu de son prochain cours  
\item Le système envoie une notification à Benoit 
\item Benoit ouvre la notification sur son smartphone 
\item L’application affiche “Le cours de Mathématiques du 27/09/2013 13h30-15h00 est annulé”  
\item Benoit sélectionne l’option “Visualiser mon  emploi du temps”  
\item L’application ouvre un document contenant l’emploi du temps mis à jour 
\end{itemize}

\subsection*{Ramener un livre emprunté}
Acteur : Bernard (Etudiant)
\begin{itemize}
\item Le système envoie une notification à Bernard 
\item Bernard ouvre la notification sur son smartphone 
\item L’application affiche “Le voyage autour du monde à rendre pour le 14/09/2013” 
\item Bernard sélectionne “rendre le livre maintenant” 
\item L’application affiche une carte indiquant le chemin à suivre pour se rendre à la bibliothèque depuis sa position actuelle 
\end{itemize}

\subsection*{Ramener un livre emprunté}
Acteur : Bernard (Etudiant)
\begin{itemize}
\item Le système envoie une notification à Bernard 
\item Bernard ouvre la notification sur son smartphone 
\item L’application affiche “Le voyage autour du monde à rendre pour le 14/09/2013” 
\item Bernard sélectionne “ne pas rendre le livre maintenant” 
\item L’application affiche “Vous ne pourrez plus emprunter de livre pendant le temps de retard + 2 semaines” 
\end{itemize}

\newpage

\subsection{Restauration}

Les utilisateurs peuvent consulter les informations sur les restaurants universitaires et décider d'activiter le guidage jusqu'au restaurant choisit.

\subsection*{Choisir un RU pour consommation instantanée avec guidage}
Acteur : Hector (Etudiant)
\begin{itemize}
\item Hector a faim 
\item Hector ouvre l’application sur son smarthone 
\item L’application affiche la liste des sections disponibles 
\item Hector sélectionne la section “Restauration” 
\item L’application affiche la liste des restaurants universitaires disponibles 
\item Hector clique sur le “restaurant universitaire Le Sully” 
\item L’application affiche une liste déroulante présentant le repas du jour, les horaires d’ouvertures et le taux d’occupation 
\item Hector clique sur le bouton "Plan"
\item L’application affiche une carte indiquant le chemin à suivre pour se rendre au restaurant universitaire Le Sully depuis sa position actuelle 
\end{itemize}

\newpage

\subsection{Plan (géo-localisation)}

La géo-localisation permet aux utilisateurs de se diriger vers un bâtiment, une salle ou un autre utilisateur. Les visiteurs ne peuvent pas recherché des personnes enregistrés dans le système.

\subsection*{Recherche d’un ami}
Acteur : Charles (Etudiant), Marcel (Etudiant)
\begin{itemize}
\item Charles ouvre l’application sur son smartphone 
\item L’application affiche la liste des sections disponibles 
\item Charles sélectionne la section “Plan” 
\item L’application affiche une carte ainsi qu’une liste d’onglets 
\item Charles sélectionne l’onglet “Retrouver un ami” 
\item L'application affiche une saisie pour le nom et prénom de l'ami recherché 
\item Charles entre le nom et le prénom de son ami Marcel 
\item Le système envoie “Charles veut vous rejoindre.  Autorisez le vous à vous localiser ?” à Marcel 
\item Marcel accepte la requête 
\item L’application lance le service de guidage 
\item Charles suit les instructions pour rejoindre Marcel  
\end{itemize}

\subsection*{Recherche d'un ami}
Acteur : Charles (Etudiant), Marcel (Etudiant)
\begin{itemize}
\item Charles ouvre l’application sur son smartphone 
\item L’application affiche la liste des sections disponibles 
\item Charles sélectionne la section “Plan” 
\item L’application affiche une carte ainsi qu’une liste d’onglets 
\item Charles sélectionne l’onglet “Retrouver un ami” 
\item L’application affiche une saisie pour le nom et prénom de l'ami recherché
\item Charles entre le nom et le prénom de son ami Marcel 
\item Le système envoie “Charles veut vous rejoindre.  Autorisez le vous à vous localiser ?” à Marcel 
\item Marcel n’entend pas qu’il a reçu une notification 
\item Au bout de cinq minute d’attente, le système envoie à Charles “Marcel ne peut pas être localisé pour le moment” 
\end{itemize}

\subsection*{Recherche d’un ami qui ne veut pas être retrouvé}
Acteur : Charles (Etudiant), Marcel (Etudiant)
\begin{itemize}
\item Charles ouvre l’application sur son smartphone 
\item Le système affiche la liste des sections disponibles 
\item Charles sélectionne la section “Plan” 
\item Le système affiche la carte ainsi qu’une liste d’onglets 
\item Charles sélectionne l’onglet “Retrouver un ami” 
\item L’application affiche une saisie pour le nom et prénom de l'ami recherché 
\item Charles entre le nom et le prénom de son ami Marcel 
\item Le système envoie “Charles veut vous rejoindre.  Autorisez le vous à vous localiser ?” à Marcel 
\item Marcel refuse la requête 
\item Le système envoie à Charles “Marcel ne veut pas être localisé”
\end{itemize}

\subsection*{Recherche de bâtiment}
Acteur: Charles (Etudiant)
\begin{itemize}
\item Charles ouvre l’application sur son smartphone 
\item L’application affiche la liste des sections disponibles 
\item Charles sélectionne la section “Plan” 
\item L’application affiche la carte ainsi qu’une liste d’onglets 
\item Charles sélectionne “M1” sur la carte 
\item L’application affiche une carte indiquant le chemin à suivre pour se rendre au bâtiment “M1” depuis sa position actuelle 
\item Charles suit les instructions pour rejoindre le bâtiment “M1” 
\end{itemize}

\subsection*{Recherche de salle}
Acteur : Charles (Etudiant)
\begin{itemize}
\item Charles ouvre l’application sur son smartphone 
\item L’application affiche la liste des sections disponibles 
\item Charles sélectionne la section “Plan” 
\item L’application affiche la carte ainsi qu’une liste d’onglets 
\item Le système détermine par géo-localisation, le bâtiment dans lequel se trouve Charles 
\item L’application affiche “Vous êtes actuellement au M1” ainsi que la liste des salle du M1 
\item Charles sélectionne “116” dans la liste des salles 
\item L’application affiche une carte indiquant le chemin à suivre pour se rendre à la salle “116” depuis sa position actuelle 
\item Charles suit les instructions pour rejoindre la salle “116” 
\end{itemize}

\newpage

\subsection{Bibliothèque}

Les utilisateurs peuvent emprunter des médias ou les consulter. Les visiteurs n'ont pas accès à ces possibilités.

\subsection*{Emprunt de livre}
Acteur : Charles (Etudiant)
\begin{itemize}
\item Charles ouvre l’application sur son smartphone 
\item L’application affiche la liste des sections disponibles 
\item Charles sélectionne la section “Bibliothèque” 
\item L’application affiche une liste d’onglets 
\item Charles sélectionne “Recherche d’un livre” 
\item L’application affiche une saisie pour le nom du livre 
\item Charles entre Le voyage autour du monde qu’il souhaite emprunter 
\item L’application affiche une liste d’informations détaillées (auteur, édition, statut, disponibilité, lieu de stockage), le livre est en format physique 
\item Charles réserve le livre  
\item Le système envoie une confirmation pour aller le chercher à la bibliothèque 
\end{itemize}

\subsection*{Emprunt de livre}
Acteur : Charles (Etudiant)
\begin{itemize}
\item Charles ouvre l’application sur son smartphone 
\item L’application affiche la liste des sections disponibles 
\item Charles sélectionne la section “Bibliothèque” 
\item L’application affiche une liste d’onglets 
\item Charles sélectionne “Recherche d’un livre” 
\item L’application affiche une saisie pour le nom du livre 
\item Charles fait une recherche sur le livre Le voyage autour du monde qu’il souhaite emprunter 
\item L’application affiche une liste d’informations détaillées (auteur, édition, statut, disponibilité, lieu de stockage), le livre est en format physique ou électronique 
\item Charles sélectionne le format électronique et appuie sur le bouton “télécharger” 
\item L’application télécharge le livre sur le smartphone qui est lisible depuis n’importe quel lecteur
\end{itemize}

\subsection*{Consulter ses emprunts}
Acteur : Charles (Etudiant)
\begin{itemize}
\item Charles veut consulter ses emprunts 
\item Charles ouvre l’application sur son smartphone 
\item L’application affiche la liste des sections disponibles 
\item Charles sélectionne la section “Bibliothèque” 
\item L’application affiche une liste d’onglets 
\item Charles sélectionne “Consulter mes emprunts” 
\item L’application affiche la liste de ses emprunts, et des livres qu’il a réservé (Date d’emprunt, Date de rendu) 
\end{itemize}

\newpage

\subsection{Événements culturels}

Les informations sur les événements culturels sont visibles par les utilisateurs, ils peuvent les rechercher selon leurs préférences.

\subsection*{Aperçu d’événement}
Acteur : Charles (Etudiant)
\begin{itemize}
\item Charles ouvre l’application sur son smartphone 
\item L’application affiche la liste des sections disponibles 
\item Charles sélectionne la section “Évènements culturels” 
\item L’application affiche la liste des évènements culturels à venir 
\item L’application vérifie les préférences ajoutées  
\item Charles n’a pas de préférence, l’application affiche la liste des événements à venir par ordre chronologique 
\end{itemize}

\subsection*{Ajout d’événements dans favoris}
Acteur : Charles (Etudiant)
\begin{itemize}
\item Charles ouvre l’application sur son smartphone 
\item L’application affiche la liste des sections disponibles 
\item Charles sélectionne “Évènements culturels” 
\item L’application vérifie les préférences ajoutées  
\item Les préférences n’ont pas été ajoutées, il les affiche de façon chronologique 
\item Charles sélectionne l’événement “Concert des NNBS” 
\item L’application demande si cet événement doit être placé dans les favoris 
\item Charles valide la proposition 
\item L’application crée une nouvelle liste des événements culturels favoris de l’étudiant 
\end{itemize}

\newpage

\subsection{Sécurité}

La sécurité concernent la fermeture des salles et la connexion au système. La fermeture des salles est automatique à 21 heures, elle n'est pas effectuée si quelqu'un se trouve dans la salle

\subsection*{Connexion à la plate forme}
Acteur : Cécile (Etudiante)
\begin{itemize}
\item Cécile lance l’application pour la première fois 
\item L’application affiche une saisie pour le login et le mot de passe 
\item Cécile entre son login et mot de passe 
\item L’application vérifie les identifiants 
\item Cécile est connectée avec les sections disponibles pour les étudiants 
\end{itemize}

\subsection*{Connexion à la plate forme avec de mauvais identifiants}
Acteur : Cécile (Etudiante)
\begin{itemize}
\item Cécile lance l’application pour la première fois 
\item L’application affiche une saisie pour le login et le mot de passe 
\item Cécile entre son login et son mot de passe 
\item L’application ne trouve pas ses identifiants dans la base de données 
\item L’application affiche “Erreur de login ou de mot de passe” 
\end{itemize}

\subsection*{Fermeture automatique d’une salle}
\begin{itemize}
\item Il est 21 heures, le système vérifie qu’il n’y a personne dans la salle grâce aux détecteurs de présence 
\item Le système ferme la salle       
\end{itemize}

\subsection*{Fermeture automatique des salles}
Acteur : Serges (Visiteur)
\begin{itemize}
\item Il est 21 heures, le système vérifie qu’il n’y a personne dans la salle aux puces sur les smartphones 
\item Le sytème vérifie grâce aux détecteurs de présence 
\item Serges est toujours dans la salle, une notification est envoyée à la sécurité du campus 
\item Le système ferme la salle en attendant une notification de la sécurité 
\end{itemize}

\subsection*{Fermeture automatique des salles avec une personne dans la salle}
Acteur : Monsieur Plip (Professeur)
\begin{itemize}
\item Il est 21 heures, le système vérifie qu’il n’y a personne dans la salle grâce aux puces sur les smartphones 
\item Il y a quelqu’un dans la salle, il envoie une notification au smartphone de Monsieur Plip 
\item Le système vérifie toutes les cinq minutes 
\item Monsieur Plip n’est toujours pas sorti au bout d’une demi-heure, le système envoie une notification aux agents de sécurité du campus 
\end{itemize}

\newpage

\section{Non Etudiant (équipe pédagogique/chercheurs)}

Le non étudiant peut accéder aux memes services que l’étudiant 

\newpage

\subsection{Emploi du temps}

Les étudiants et professeurs peuvent consulter leurs emplois du temps. Les secrétaires ont l'autorisation de consulter et modifier l'emploi du temps de tous les professeurs et groupes d'élèves. Les professeurs modifient uniquement leur emploi.

\subsection*{Regarder l’emploi du temps}
Acteur : Bernard (Etudiant)
\begin{itemize}
\item Bernard ouvre l’application sur son smartphone 
\item L’application affiche la liste des sections disponibles 
\item Bernard sélectionne la section “Emploi Du Temps” 
\item L’application affiche une liste d’onglets et l’emploi du temps mis à jour 
\end{itemize}

\subsection*{Changement d’horaire}
Acteur : Monsieur Plip (Professeur)
\begin{itemize}
\item Monsieur Plip ouvre l’application sur son smartphone 
\item L’application affiche la liste des sections disponibles 
\item Charles sélectionne la section “Emploi Du Temps” 
\item L’application affiche une liste d’onglets et l’emploi du temps mis à jour 
\item Monsieur Plip sélectionne l’onglet “Modification d’horaire” 
\item L’application affiche une saisie de la date et de l’heure de début 
\item Monsieur Plip saisit le “05/10” à ”8h00” 
\item L’application affiche une saisie pour le jour et l’heure de début du cours à changer 
\item Monsieur Plip sélectionne de déplacer son cours pour le “11/10” à “10h00” 
\item L’application vérifie la disponibilité et valide la modification 
\item L’application prévient les étudiants concernés ainsi que le secrétariat 
\end{itemize}

\subsection*{Changement de salle}
Acteur : Monsieur Plip (Professeur)
\begin{itemize}
\item Monsieur Plip ouvre l’application sur son smartphone 
\item L’application affiche la liste des sections disponibles 
\item Charles sélectionne la section “Emploi Du Temps” 
\item L’application affiche une liste d’onglets et l’emploi du temps mis à jour 
\item Monsieur Plip sélectionne l’onglet “Modification de salle” 
\item L’application affiche une saisie pour le jour et l’heure de début de cours 
\item Monsieur Plip saisit le “05/10” à ”8h00” 
\item L’application affiche une saisie pour la nouvelle salle 
\item Monsieur Plip sélectionne de déplacer son cours en “A116” 
\item L’application vérifie la disponibilité et valide la modification 
\item Le système notifie les étudiants concernés ainsi que le secrétariat 
\end{itemize}

\subsection*{Changement de salle pour une salle prise}
Acteur : Monsieur Plip (Professeur)
\begin{itemize}
\item Monsieur Plip ouvre l’application sur son smartphone 
\item L’application affiche la liste des sections disponibles 
\item Charles sélectionne la section “Emploi Du Temps” 
\item L’application affiche une liste d’onglets et l’emploi du temps mis à jour 
\item Monsieur Plip sélectionne l’onglet “Modification de salle” 
\item L’application affiche une saisie pour le jour et l’heure de début de cours 
\item Monsieur Plip saisit le “05/10” à ”8h00” 
\item L’application affiche une saisie pour la nouvelle salle 
\item Monsieur Plip sélectionne de déplacer son cours en “A116” 
\item L’application vérifie la disponibilité et refuse la modification 
\item L’application affiche un pop-up indiquant la raison du refus 
\end{itemize}

\newpage

\subsection{Actions sur les bâtiments}

Les professeurs et certains étudiants peuvent ouvrir des salles. Il est possible pour eux de modifier la température

\subsection*{Ouverture de salle}
Acteur : Monsieur Plip (Professeur)
\begin{itemize}
\item Monsieur Plip passe son badge devant la porte 
\item Le système vérifie le profil du badge 
\item Le système reçoit le profil professeur 
\item Le système déverrouille la salle 
\end{itemize}

\subsection*{Modification de température}
Acteur : Madame Trombone (Secrétaire)
\begin{itemize}
\item Madame Trombone veut modifier la température de son bureau 
\item Madame Trombone ouvre l’application sur son smartphone 
\item L’application affiche la liste des sections disponibles 
\item Madame Trombone sélectionne “Température” 
\item L‘application affiche une saisie pour la température souhaitée 
\item Madame Trombone saisit "26" 
\item Le système enregistre la modification 
\item La salle a une température de 26 degrès 
\end{itemize}

\newpage

\subsection{Chauffage}

Le chaufage s'allume dans une pièce 30 minutes avant qu'elle soit utilisée. Si les horaires d'utilisation de la pièce ont changé, la routine change automatiquement.

\subsection*{Routine chauffage}
\begin{itemize}
\item Le système lance la routine de gestion de chauffage. Il consulte la base de donnée pour connaître les salles utilisées pour la journée 
\item Le système évalue la température et commence à chauffer les salles jusqu’à une température acceptable, une heure avant que les salles soient occupées 
\end{itemize}

\subsection*{Modification d’horaire}
\begin{itemize}
\item Le système reçoit une notification sur la modification de l’EDT (l’attribution des salles) 
\item Le système relance la vérification sur les salles dont le statut est modifiées 
\item Une salle est chauffée alors qu’il n’y aura personne dedans la prochaine heure 
\item Le chauffage/La climatisation de la salle s’arrête 
\end{itemize}

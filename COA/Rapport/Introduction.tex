\chapter{Introduction}

Dans le cadre du projet de COA nous avons choisi le sujet: Université du futur.
Notre objectif est de créer les éléments nécessaires à la réalisation d'un tel projet.

\section{Sujet}

On se propose de concevoir le nouveau système informatique de l’Université Lille 1.
Ce système doit gérer la sécurité (accès par badge) et l’énergie (chauffage) des bâtiments.
Ce système doit aussi gérer l’accès aux services aux étudiants (via leur smartphone): 
\begin{itemize}
\item emploi du temps
\item notifications (absence d'un enseignant, changement de salle ...)
\item examens (rappels)
\item géo-localisation et guidage (vers les amis, les restaurants, les salles de cours)
\item bibliothèques (emprunts, disponibilité d’un livre désiré, rappel lors de retard)
\item repas au restaurant universitaire (affichage des restaurants ouverts et de leurs menus, triés en fonction des préférences de l’étudiant)
\item évènements culturels 
\end{itemize}

\section{Limite du système}

Le système que nous élaborons ne tient pas compte des parties matérielles. De plus l'application est visible aussi bien depuis un smartphone que via un ordinateur. L'application mobile ou le rendu web n'ont pour but que de récupérer les informations depuis le système.

Certaines fonctionnalités seront disponibles sans compte pour permettre à de futurs étudiants de s'inscrire ou encore de se déplacer sur le campus sans se perdre.
Voici une liste des fonctionnalités qui pour l'instant seront accesibles sans authentification:

\begin{itemize}
\item inscription
\item plan
\item repas
\item bibliothèque (sans la possibilité de réserver/emprunter)
\end{itemize}

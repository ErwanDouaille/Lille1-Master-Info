\chapter{Contexte}

Ce projet s'insère dans une collaboration entre l'équipe MINT du LIFL et la Fondation Hopale, visant à mettre à disposition du Centre Jacques Calvé (à Berck-sur-Mer) des applications utilisant les technologies de la Réalité Virtuelle dans un but de rééducation fonctionnelle.

\section{Fondation Hopale}

La Fondation Hopale est un établissement reconnu d'utilité publique. Elle représente 1100 lits et places dans la région Nord/Pas-de-Calais et un effectif de 2500 personnes dans les secteurs sanitaire ESPIC (Etablissement de Santé Privé d'Intérêt Collectif) et Médico-social. Son activité hospitalière est essentiellement orientée vers les pathologies de l’appareil locomoteur, tant en court séjour (chirurgie orthopédique, rhumatologie, neurologie, prise en charge de la douleur), qu’en soins de suite et de réadaptation (département des blessés crâniens, traumatisés médullaires, polytraumatisés et autres prises en charge spécialisées lourdes). Son activité de rééducation dispose avec le Centre Jacques Calvé du plus grand plateau technique de France.

\section{Équipe MINT LIFL}

L'équipe MINT travaille sur l'interaction homme machine. L'interaction gestuelle est leurs sujet d'étude principal, c'est à dire l'utilisation du geste dans une interaction homme machine. Le geste est définit comme un mouvement d'une partie du corps pour exprimer une idée ou un sens, souvent réalisé avec les mains ou la tête. Cette interaction peut se traduire par la pression d'un doigt sur une surface tactile ou encore une information de position.

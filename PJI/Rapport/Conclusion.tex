\chapter{Conclusion}
Le projet, consistant à rendre l'utilisation d'une application de dessin à main levée, avec une caméra Kinect, a été finalisé. Pour cela, nous avons du ajouter un tracker qui récupère le squelette de l'utilisateur puis filtrer les positions pour avoir un signal net. Cette application servira pour faire de la rééducation fonctionelle dans la fondation Hopale.

Ce projet nous a permis d'ameliorer nos connaissances techniques. En effet, nous avons du apprendre à utiliser OpenNI pour récuperer et manipuler un squelette detecté avec la caméra Kinect. En plus de ces compétences liées à la Kinect, nous avons manipulé le filtre \textit{One Euro Filter}, ce qui nous a permis de comprendre comment pallier les problèmes de latence liés à des périphériques de tracking.

Une introduction à la librairie Clutter nous a également permis de comprendre et analyser, comment fonctionne WabinPaint. 
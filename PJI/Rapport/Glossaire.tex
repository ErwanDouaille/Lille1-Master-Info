\chapter{Glossaire}

\textbf{Kinect} Il s'agit d'une caméra utilisant des techniques d'interaction
développées par la société israélienne
PrimeSense, longtemps nommée par
le nom de code du projet, "Project Natal", elle a été officialisée juste avant l'E3 sous le nom Kinect. Elle est basée sur un périphérique d'entrée
branché sur la console Xbox 360 qui permet d'interagir par commande vocale
(pas disponible au lancement en France), reconnaissance de mouvement et
d'image.\\

\textbf{Open Sound Control} (OSC) est un format de transmission de données
entre ordinateurs, synthétiseurs, robots ou tout autre matériel ou logiciel
compatible, conçu pour le contrôle en temps réel. Il utilise le réseau au
travers des protocoles UDP ou TCP.\\

\textbf{OpenNi} le framework OpenNi fournit des API open source. Ces API tendent à devenir des standards pour les applications ayant accés à des
dispositifs d'interaction naturelle, tel que la Kinect. Ces API fournissent
une reconnaissance de voix, du tracking de corps, détection de mouvements,
gestes, etc.\\


\textbf{Tracking} est l'ensemble des moyens mis en oeuvre pour suivre, dans
notre contexte, un corps en mouvement.\\

\textbf{WabinPaint} est une application de dessin développée par Samuel
Degrande, Ingénieur de Recherche au LIFL et membre de l'équipe-projet
commune Mint.\\